%% This file is encoded in utf-8
%%
\documentclass[12pt,landscape,english]{article}
\usepackage{babel}
\usepackage{multicol}
\usepackage{calc}
\usepackage{ifthen}
\usepackage[landscape]{geometry}
\usepackage{amsmath,amsthm,amsfonts,amssymb}
\usepackage{color,graphicx,overpic}
\usepackage{hyperref}

\usepackage[T1]{fontenc}
\usepackage[utf8]{inputenc}
\usepackage[defaultsans]{droidsans}
\renewcommand*\familydefault{\sfdefault}

\pdfinfo{
  /Title (Debian Bookworm GNOME sneltoetsen)
  /Creator (TeX)
  /Producer (pdfTeX 1.40.0)
  /Author (Steven Speek)
  /Subject (Example)
  /Keywords (pdflatex, latex,pdftex,tex)}

% This sets page margins to .5 inch if using letter paper, and to 1cm
% if using A4 paper. (This probably isn't strictly necessary.)
% If using another size paper, use default 1cm margins.
\ifthenelse{\lengthtest { \paperwidth = 11in}}
    { \geometry{top=.5in,left=.5in,right=.5in,bottom=.5in} }
    {\ifthenelse{ \lengthtest{ \paperwidth = 297mm}}
        {\geometry{top=1cm,left=1cm,right=1cm,bottom=1cm} }
        {\geometry{top=1cm,left=1cm,right=1cm,bottom=1cm} }
    }

% Turn off header and footer
\pagestyle{empty}

% Redefine section commands to use less space
\makeatletter
\renewcommand{\section}{\@startsection{section}{1}{0mm}%
                                {-1ex plus -.5ex minus -.2ex}%
                                {0.5ex plus .2ex}%x
                                {\normalfont\large\bfseries}}
\renewcommand{\subsection}{\@startsection{subsection}{2}{0mm}%
                                {-1explus -.5ex minus -.2ex}%
                                {0.5ex plus .2ex}%
                                {\normalfont\normalsize\bfseries}}
\renewcommand{\subsubsection}{\@startsection{subsubsection}{3}{0mm}%
                                {-1ex plus -.5ex minus -.2ex}%
                                {1ex plus .2ex}%
                                {\normalfont\small\bfseries}}
\makeatother

% Define BibTeX command
\def\BibTeX{{\rm B\kern-.05em{\sc i\kern-.025em b}\kern-.08em
    T\kern-.1667em\lower.7ex\hbox{E}\kern-.125emX}}

% Don't print section numbers
\setcounter{secnumdepth}{0}


\setlength{\parindent}{0pt}
\setlength{\parskip}{0pt plus 0.5ex}

%My Environments
%\newtheorem{example}[section]{Example}
% -----------------------------------------------------------------------

% Shortcut commands
\newcommand{\acc}[1]{\texttt{#1}}

\newcommand{\super}[1]{\acc{SUPER+#1}}

\newcommand{\control}[1]{\acc{CTRL+#1}}

\newcommand{\sneltoetsrij}[2]{#1 & #2 \\}

\newcommand{\supertoetsdef}[2]{\sneltoetsrij{\super{#1}}{#2}}

\newcommand{\sneltoetsdef}[2]{\sneltoetsrij{\acc{#1}}{#2}}

\newcommand{\dubbeledef}[3]{\sneltoetsdef{#1}{}\sneltoetsdef{#2}{#3}}

\newcommand{\controldef}[2]{\sneltoetsrij{\control{#1}}{#2}}


\begin{document}
\raggedright
\footnotesize
\begin{multicols}{3}

% multicol parameters
% These lengths are set only within the two main columns
%\setlength{\columnseprule}{0.25pt}
\setlength{\premulticols}{1pt}
\setlength{\postmulticols}{1pt}
\setlength{\multicolsep}{1pt}
\setlength{\columnsep}{2pt}
{
% title
\Large Debian Bookworm \linebreak GNOME sneltoetsen 
\footnote{Door Steven Speek --- \url{http://github.com/slspeek/debian-gnome-sneltoetsen}}
\rule{\columnwidth}{0.11pt}
}

\section{Algemeen}

\subsection{Widgets}
\begin{tabular}{@{}l@{\;}l@{}}
	\sneltoetsdef{SPATIEBALK}{Knop/item/optie activeren}
	\sneltoetsdef{ALT+↓}{Dropdown lijst openen}
	\sneltoetsdef{TAB}{Focus volgend widget}
	\sneltoetsdef{SHIFT+TAB}{Focus vorig widget}
	\sneltoetsdef{F10}{Menubalk activeren}
	\sneltoetsdef{MENU}{Context menu openen}
\end{tabular}

\subsection{Veelvoorkomende functies}
\begin{tabular}{@{}l@{\;}l@{}}
	\controldef{C}{Kop\"{i}eren}
	\controldef{X}{Knippen}
	\controldef{V}{Plakken}
	\controldef{F}{Zoeken}
\end{tabular}

\section{GNOME}

\subsection{Starters}
\begin{tabular}{@{}l@{\;}l@{}}
	\supertoetsdef{1}{Firefox}
	\supertoetsdef{2}{Evolution (email)}
	\supertoetsdef{3}{LibreOffice writer}
	\dubbeledef{SUPER+4}{SUPER+E}{Bestandsbeheerder}
	\supertoetsdef{5}{Software}
	\supertoetsdef{F1}{Hulp openen}
	\sneltoetsdef{CTRL+ALT+T}{Terminal}
	\supertoetsdef{L}{Scherm vergrendelen}
\end{tabular}

\subsection{Wisselen}
\begin{tabular}{@{}l@{\;}l@{}}
	\dubbeledef{SUPER+TAB}{ALT+TAB}{Wisselen tussen open toepassingen}
\end{tabular}

\subsection{Venster}
\begin{tabular}{@{}l@{\;}l@{}}
	\supertoetsdef{↑}{Venster maximaliseren}
	\supertoetsdef{↓}{Venster herstellen}
	\supertoetsdef{F10}{Maximaliseren/herstellen}
\end{tabular}

\end{multicols}

\end{document}
