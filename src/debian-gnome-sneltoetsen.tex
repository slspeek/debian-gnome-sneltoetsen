%% This file is encoded in utf-8
%%
\documentclass[12pt,landscape,english]{article}
\usepackage{babel}
\usepackage{multicol}
\usepackage{calc}
\usepackage{ifthen}
\usepackage[landscape]{geometry}
\usepackage{amsmath,amsthm,amsfonts,amssymb}
\usepackage{color,graphicx,overpic}
\usepackage{hyperref}

\usepackage[T1]{fontenc}
\usepackage[utf8]{inputenc}
\usepackage[defaultsans]{droidsans}
\renewcommand*\familydefault{\sfdefault}

\pdfinfo{
  /Title (Debian Bookworm GNOME sneltoetsen)
  /Creator (TeX)
  /Producer (pdfTeX 3.141592653-2.6-1.40.24)
  /Author (Steven Speek)
  /Subject (Sneltoetsen Debian Trixie)
  /Keywords (sneltoetsen, Debian, GNOME)}

% This sets page margins to .5 inch if using letter paper, and to 1cm
% if using A4 paper. (This probably isn't strictly necessary.)
% If using another size paper, use default 1cm margins.
\ifthenelse{\lengthtest { \paperwidth = 11in}}
    { \geometry{top=.5in,left=.5in,right=.5in,bottom=.5in} }
    {\ifthenelse{ \lengthtest{ \paperwidth = 297mm}}
        {\geometry{top=1cm,left=1cm,right=1cm,bottom=1cm} }
        {\geometry{top=1cm,left=1cm,right=1cm,bottom=1cm} }
    }

% Turn off header and footer
\pagestyle{empty}

% Redefine section commands to use less space
\makeatletter
\renewcommand{\section}{\@startsection{section}{1}{0mm}%
                                {-1ex plus -.5ex minus -.2ex}%
                                {0.5ex plus .2ex}%x
                                {\normalfont\large\bfseries}}
\renewcommand{\subsection}{\@startsection{subsection}{2}{0mm}%
                                {-1explus -.5ex minus -.2ex}%
                                {0.5ex plus .2ex}%
                                {\normalfont\normalsize\bfseries}}
\renewcommand{\subsubsection}{\@startsection{subsubsection}{3}{0mm}%
                                {-1ex plus -.5ex minus -.2ex}%
                                {1ex plus .2ex}%
                                {\normalfont\small\bfseries}}
\makeatother

% Define BibTeX command
\def\BibTeX{{\rm B\kern-.05em{\sc i\kern-.025em b}\kern-.08em
    T\kern-.1667em\lower.7ex\hbox{E}\kern-.125emX}}

% Don't print section numbers
\setcounter{secnumdepth}{0}


\setlength{\parindent}{0pt}
\setlength{\parskip}{0pt plus 0.5ex}

% Shortcut commands
\newcommand{\acc}[1]{\texttt{#1}}

\newcommand{\super}[1]{\acc{SUPER+#1}}

\newcommand{\control}[1]{\acc{CTRL+#1}}

\newcommand{\sterretje}[0]{$^*$}

\newcommand{\dubbelsterretje}[0]{$^{**}$}

\newcommand{\sneltoetsrij}[2]{#1 & #2 \\}

\newcommand{\supertoetsdef}[2]{\sneltoetsrij{\super{#1}}{#2}}

\newcommand{\sneltoetsdef}[2]{\sneltoetsrij{\acc{#1}}{#2}}

\newcommand{\dubbeledef}[3]{\sneltoetsdef{#1}{}\sneltoetsdef{#2}{#3}}

\newcommand{\controldef}[2]{\sneltoetsrij{\control{#1}}{#2}}

\newenvironment{sneltoetsen}{\begin{tabular}{@{}l@{\;}l@{}}}{\end{tabular}}

\begin{document}
\raggedright
\footnotesize
\begin{multicols}{3}

% multicol parameters
% These lengths are set only within the two main columns
%\setlength{\columnseprule}{0.25pt}
\setlength{\premulticols}{1pt}
\setlength{\postmulticols}{1pt}
\setlength{\multicolsep}{1pt}
\setlength{\columnsep}{2pt}
{
% title
\large{Debian GNOME 48.4 sneltoetsen }
%  \linebreak
% \normalsize Steven Speek\linebreak
% \footnotesize \url{http://github.com/slspeek/debian-gnome-sneltoetsen}
\rule{\columnwidth}{0.11pt}
}

\subsection{Algemeen}
\begin{sneltoetsen}
	\sneltoetsdef{SPATIEBALK}{Knop/item/optie activeren}
	\sneltoetsdef{ALT+↓}{Dropdown lijst openen}
	\sneltoetsdef{TAB}{Focus volgend widget}
	\sneltoetsdef{SHIFT+TAB}{Focus vorig widget}
	\sneltoetsdef{F10}{Menubalk activeren}
	\sneltoetsdef{F11}{Fullscreen}
	\dubbeledef{MENU}{SHIFT+F10}{Context menu openen}
	\sneltoetsdef{ESC}{Menu/Popup sluiten, annuleren}
	\controldef{C}{Kopi\"{e}ren}
	\controldef{X}{Knippen}
	\controldef{V}{Plakken}
	\controldef{S}{Opslaan}
	\controldef{O}{Openen}
	\controldef{P}{Printen}
	\controldef{Z}{Undo (terugdraaien)}
	\controldef{Y}{Opnieuw}
	\controldef{F}{Zoeken}
	\controldef{N}{Nieuw venster}
	\controldef{Q}{Afsluiten (alle vensters)}
	\controldef{-}{Uitzoomen}
	\controldef{SHIFT+'+'}{Inzoomen}
	\controldef{0}{Standaard grootte}
	\sneltoetsdef{F1}{Hulp}
	\sneltoetsdef{CTRL+SHIFT+?}{Sneltoetsen}
\end{sneltoetsen}

\subsection{Starters}
\begin{sneltoetsen}
	\supertoetsdef{1}{Firefox}
	\supertoetsdef{2}{Evolution (email)}
	\dubbeledef{SUPER+3}{SUPER+E \sterretje}{Bestandsbeheerder}
	\supertoetsdef{4}{Software}
	\supertoetsdef{5}{Teksteditor}
	\supertoetsdef{6}{Calculator}
	\dubbeledef{\super{F1}}{\super{7}}{Hulp openen}
	\sneltoetsdef{CTRL+ALT+T \sterretje}{Terminal}
	\supertoetsdef{L}{Scherm vergrendelen}
	\supertoetsdef{I \sterretje}{Instellingen}
	\sneltoetsdef{SHIFT+SUPER+I \sterretje}{Afstellingen}
\end{sneltoetsen}

\subsection{Schermafdrukken}
\begin{sneltoetsen}
	\sneltoetsdef{ALT+PrintScreen}{Schermafdruk actieve venster}
	\sneltoetsdef{PrintScreen}{Schermafdruk (interactief)}
	\sneltoetsdef{CTRL+ALT+SHIFT+R}{Screencast opnemen}
\end{sneltoetsen}

\subsection{Wisselen}
\begin{sneltoetsen}
	\sneltoetsdef{SUPER}{Activiteiten modus}
	\supertoetsdef{A}{Applicatie modus}
	\supertoetsdef{V}{Notificaties/Kalender}
	\supertoetsdef{N}{Focus actieve notificatie}
	\dubbeledef{SUPER+TAB}{ALT+TAB}{Wisselen tussen toepassingen}
	\sneltoetsdef{ALT+F6}{Snel wisselen 1 toepassing}
	\sneltoetsdef{ALT+\`{}}{Venster wisselen 1 toepassing}
	\sneltoetsdef{ALT+ESC}{Snel wisselen vensters 1 werkblad}
	\supertoetsdef{F10}{Toepassings menu (bovenbalk)}
	\sneltoetsdef{CTRL+ALT+TAB}{Bovenblak/Vensters}
	\supertoetsdef{Home}{Eerste werkblad}
	\supertoetsdef{PgDn}{Volgend werkblad}
	\sneltoetsdef{SHIFT+SUPER+PgDn}{Venster mee volgend werkblad}
	\supertoetsdef{PgUp}{Vorig werkblad}
	\sneltoetsdef{SHIFT+SUPER+PgUp}{Venster mee vorig werkblad}
	\supertoetsdef{End}{Laatste werkblad}
	\supertoetsdef{L}{Scherm vergrendelen}
	\sneltoetsdef{CTRL+ALT+DELETE \sterretje}{Uitschakelen}
\end{sneltoetsen} 

\subsection{Venster}
\begin{sneltoetsen}
	\sneltoetsdef{ALT+SPATIEBALK}{Venstermenu}
	\sneltoetsdef{ALT+F4}{Sluiten}
	\supertoetsdef{↑}{Venster maximaliseren}
	\supertoetsdef{↓}{Venster herstellen}
	\supertoetsdef{←}{Linker helft scherm}
	\supertoetsdef{→}{Rechter helft scherm}
	\sneltoetsdef{ALT+F10}{Maximaliseren/herstellen}
	\supertoetsdef{H}{Verbergen}
\end{sneltoetsen}
\subsubsection{Verplaatsen/Grootte}
\begin{sneltoetsen}
	\sneltoetsdef{ALT+F7}{Verplaats modus}
	\sneltoetsdef{ALT+F8}{Grootte modus}
	\sneltoetsdef{CTRL+←↑↓→}{Per pixel}
	\sneltoetsdef{←↑↓→}{Verplaatsen}
	\sneltoetsdef{SHIFT+←↑↓→}{Grote stap/naar de rand}
	\sneltoetsdef{ENTER}{Bevestigen}
	\sneltoetsdef{ESC}{Annuleren}
\end{sneltoetsen}

\subsection{Firefox}
\begin{sneltoetsen}
	\controldef{T}{Nieuw tabblad}
	\controldef{W}{Tabblad sluiten}
	\sneltoetsdef{CTRL+SHIFT+T}{Tabblad heropenen}
	\controldef{TAB}{Tabblad wisselen}
	\controldef{K}{Zoeken met zoekmachine}
	\controldef{L}{Focus naar adresbalk}
	\controldef{D}{Bladwijzer maken}
	\controldef{SHIFT+O}{Bladwijzers beheren}
	\controldef{SHIFT+B}{Bladwijzerbalk tonen/verbergen}
	\sneltoetsdef{ALT+←}{Terug}
	\sneltoetsdef{ALT+→}{Vooruit}
	\controldef{Click}{Open link in nieuw tabblad}
	\controldef{SHIFT+Click}{Ga naar link in nieuw tabblad}
	\controldef{H}{Geschiedenies tonen/verbergen}
\end{sneltoetsen}

\subsection{Bestandsbeheerder}
\begin{sneltoetsen}
	\controldef{1}{Lijstweergave}
	\controldef{2}{Rasterweergave}
	\controldef{H}{Verborgen bestanden}
	\controldef{SHIFT+N}{Nieuwe map}
	\controldef{N}{Nieuw venster}
	\controldef{W}{Venster sluiten}
	\sneltoetsdef{ALT+↑}{Naar bovenliggende map}
	\sneltoetsdef{DELETE}{Naar prullenbak verplaatsen}
	\sneltoetsdef{ALT+ENTER}{Eigenschappen}
\end{sneltoetsen}

\subsection{Speciale tekens \dubbelsterretje}
\begin{sneltoetsen}
	\sneltoetsdef{RightALT = c}{€}
	\sneltoetsdef{RightALT \'{} e}{\'{e} (acute)}
	\sneltoetsdef{RightALT \`{} e}{\`{e} (grave)}
	\sneltoetsdef{RightALT \"{} e}{\"{e} (trema)}
	\sneltoetsdef{RightALT - e}{\={e} (macron)}
	\sneltoetsdef{RightALT \^{} e}{\^{e} (circumflex)}
	\sneltoetsdef{SUPER sym ENTER}{Tekens en symbolen}
\end{sneltoetsen}

\sterretje Aanbevolen sneltoets die onder Debian niet standaard zo is ingesteld. 

\dubbelsterretje Hier is aangenomen dat de samensteltoets op de rechter alt-toets is insteld.

\end{multicols}
\end{document}
